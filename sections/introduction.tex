\section{Introduction and framework}

% not too long introduction with emphasis on locating this specific research between the relevant fields

In the past 60 years the development of new technology has fundamentally transformed music creation and consumption (\cite[\todo{what page?}]{winkler01}).
Interactive music systems is one of the fields that emerged from those transformations, facilitating new ways of music creation and blurring the traditionally implied distinction between instrument design, composition and performance (\cite{drummond09}).
Until recently, interactive music systems was intended almost solely to professional musicians (with interactive sound installations as exception).
But today, one can find many projects trying to expose those idea, of interactive music, to the average user.

Another key concept with high relevance to my work is Augmented Reality (AR). According to Azuma, AR systems integrate 3-D virtual objects into a 3-D real environment in real time (\cite*{azuma97}).
Given this necessarily visual definition the existence of ``Audio-Only'' AR system is unclear.
Later in his article Azuma states that ``AR enhances a user's perception of and interaction with the real world'', a statement which lighten the subject in much broader manner and open the door to enhance ones perception of and interaction with the world with much more than just visual information.

In this work I will focus on the new possibilities that real-time social interaction, with the help of modern mobile technology and AR as perceived by Azuma's later definition, can offer to music consumption.