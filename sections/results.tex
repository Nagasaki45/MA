\section{Initial results}

This chapter will show the initial results for both the system development as well as for the system evaluation.

\subsection{System development}

The system development could be described by two different processes, the development of the Bluetooth Based Relative Indoor Positioning (BBRIP) system and the Android application that wraps it and is responsible for the audio processing.

\subsubsection{Implementation of the BBRIP system}

The BBRIP system is my intent to develop an indoor positioning system that will satisfy the relatively simple requirements of the research:
\begin{itemize}
	\item To eliminate the use of infrastructure.
	\item To be able to estimate, roughly, the distance of each participant from his surrounding beacons.
\end{itemize}

As I've already noted in chapter \ref{methods:ips}, such a solution not yet exist.

My implementation of the system is based on a specific element in the Bluetooth protocol -- the Received Signal Strength Indicator (RSSI) (\todo{ref}). Each Bluetooth enabled device calculate RSSI values during Bluetooth discovery, when it finds a new device and before establishing connection. Thereby, the BBRIP system continuously search for Bluetooth devices. When a new device is found the RSSI value is extracted and send forward for processing. From the first discovery in the Bluetooth discovery cycle the system checks if the time since the last discovery exceeded a pre-defined timeout and if so it terminate the discovery. This termination is important because naturally a device can only be discovered once in each Bluetooth discovery cycle and long period of time without new discoveries indicates that all of the nearby devices are found. Lastly, when the system sees that there is no Bluetooth discovery running (because of termination or simply the end of the cycle) it starts a new one immediately.

Although the RSSI values extracted by the BBRIP system are not very precise as a distance estimation, I have found them sufficient enough in order to classify the distance between participants and beacons into useful ranges. In other words, RSSI values gives great indication if a participant is stands close to specific beacon (around 1 meter), in mediate range (2 to 3 meters) or in larger distance.

\subsubsection{ScenePlayer Plus}

The development of the Android application moved through different development stages.

First, I have developed an application that didn't use ``libpd'' at all (see chapter \ref{methods:libpd}). In this early stage implementation the only effect of getting close to, or far from a beacon was by changing the volume. In addition, the limited sophistication of the built in audio library made fading in and out from the sound zones very non-flexible.

After finding the weakness of using the built in audio library of the Android API I have decided to implement the system using ``libpd''. Although this implementation worked fine, it made a very hard coupling between the system development and the audio processing development in Pd. Overall, this development phase was sufficiant enough but in order to allow other musicians and developers to use the system I have started to look for more open architecture, that will maintain loosely coupled connection between the system itself and the Pd patch that drove the audio.

The last phase of in the development of the system was to implement the BBRIP system into the open source Android application ``ScenePlayer'' (), an Android port for the RjDj application mentioned in chapter \todo{add cross ref}, and released it again as ``ScenePlayer Plus'' (). The application, which can be downloaded from \href{}{Google play} and found on \href{}{Github}, expose the Bluetooth RSSI values to the Pd patch as another sensor of the mobile device (e.g. accelerometer, compass and touchscreen).

\subsection{System evaluation}

In order to evaluate our system we invited eighteen volunteers to participate in an interactive silent rave party.

Each participant installed the Android application on his or her phone and filled pre/post party surveys that included questions regarding their musical background and preferences as well as system evaluation feedback. The participants could freely move the six Bluetooth beacons which were installed on colored balloons. The party consisted of four alternating interactive/control blocks of duration 5:40 minutes each (see figure 2). The participants were randomly assigned to two groups: A and B, comprising the interactive and control blocks respectively\footnote{Group A (interactive first) consists of 8 participants (4 females and 4 males) with mean age of 36.7 (s.d=12.3); group B (control first) consists of 10 participants (3 females and 7 males) with mean age of 29.6 (s.d=10.2). Participants had a diverse musical background with 4.7 mean years of musical training (s.d=5.2).}. They were generally informed that the experiment consists of interactive and control segments, however they were not informed about the exact schedule and timing of the blocks or the group assignments. Both groups started the experiment together. In the interactive blocks, the application generated music as described above, whereas in the control blocks the participants heard recorded non-interactive music created in advance using the musical material of the interactive system\footnote{The control block music composed by Noam Elron (\href{http://www.noamelron.com}{www.noamelron.com}).}.

Interaction with the system’s components was assessed by counting the number of Bluetooth device discoveries made by each participant’s phone during both the interactive and the control blocks. In order to eliminate edge effects, we analyzed only the two middle blocks of the experiment.