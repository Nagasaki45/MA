\section{Literature review}

\subsection{The origins of IMS}

According to the Oxford Dictionary to ``interact'' is to ``act in such a way as to have an effect on each other'' (\cite{web:oxford}).
In the field of IMS actions and affects may be implemented on a broad spectrum of novel techniques, ranging from interactive sound installations to collaborations with robotic performers (\cite{drummond09}).

Since the beginning of the exploration in the field of IMS in the nineteen-sixties, different researchers and composers created systems that were designed to interact with performer in a live situation.
First example of this kind of interactive system was Gordon Mumma's Hornpipe, where specially designed electronic system alters the sound of the audio input from the performer and creates an interactive loop between the player and the sound created by the electronics circuit (\cite[page 12]{winkler01}).

Later, during the nineteen-seventies, musicians and researchers were able to create IMS programmatically using newly developed programming languages designed for musical applications, such as GROOVE or the MUSIC-N series (\cite{mathews70}; \cite{mathews69}).

In the nineteen-eighties, a group of musical instruments manufacturers agreed on universal standard for sending and receiving musical information digitally, establishing the MIDI protocol (\cite{web:quinn}).
The standardization of sending and receiving information combined with the emergence of personal computers enabled the creation of modern programing languages for musical applications.
Max/MSP, which began its development by Miller Puckette in 1986, may be a good example (\cite[page 16]{winkler01}).
As opposed to early programming languages as GROOVE or the MUSIC-N series, most of those personal computer based languages still exist today, keep progressing and new music oriented programing languages are developed on a regular basis (\cite{web:chuck}; \cite{web:usine}).

In the same time, similar technological shifts also facilitated the usage of Digital Audio Workstations (DAW) as a common alternative to analog recording equipment in studios (\todo{find ref}). Later, with the arrival of the VST standard (\cite{web:steinberg}), computers became even more essential tool for music production.
The arrival of those technologies to the stage could be identified with the emergence of live performance oriented DAWs as Ableton Live (\cite{web:live}) and software based DJ setups in late nineteen-nineties.

Another key event in the history of IMS is the appearance of the Arduino platform in 2006.
The Arduino is an easy to use hardware and software package, ``intended for artists, designers, hobbyists and anyone interested in creating interactive objects or environments'' (\cite{web:arduino}).
Until its appearance the only way for a musician to interact with music creation software was through audio and MIDI.
But from then a lot of new ways of controlling audio started to be possible, mostly by translating physical properties of the space into sound.

\subsection{Interactive music systems for non-professional musicians}

Today, IMS meets the non-professional musician in various scenarios: interactive video clips, mobile applications, interactive sound installations and social DJing.
While these examples are typical they are only a small portion of novel ways where non professional can now participate in interactive music creation, interactive consumption of audiovisual content and musically enhanced social interactions.

% interactive video clips: Interlude, Chris Milk and Aaron Koblin
Interactive video clips expose some of the roles traditionally kept for the director to the viewers.
As a relatively new phenomena video clips like those are rare but prominent.
As examples one can find the works of Chris Milk and Aaron Koblin (\cite{web:milk1}; \cite{web:milk2}) and the startup Interlude which driven by equivalent concept (\cite{web:interlude}).

% mobile applications: Smule and RjDj
The increased computation power of the modern mobile phone led developers to develop new mobile applications for music creation for non-professional musicians.
Those applications gives the average user the ability to create music by himself and without musical education.
Good examples for this kind of applications are Smule's applications, with emphasis on AutoRap (\cite{web:autorap}).
Similarly, RjDj uses the modern phones sensors to create ambient sonification that based on the users' interactions with the daily environment (\cite{web:rjdj})\label{rjdj}.

% interactive sound installations: objects with sound, project ADA
Sound installations are installations located in the three dimensional space that dialog with their surroundings through sound (``\citetitle{wiki:soundinstallation}'').
Whereas in some interactive sound installations the main interaction is between the viewer and the installation itself (\cite{web:visnjic}\todo{more refs}) there are installations that aims to engage the participants to interact with one another (\cite{eng03}).
One may say that the main goal of this kind of interaction is to facilitate social interaction\todo{find ref, maybe Omer Golan}.

% social DJing: DistributedDJ, the BLOB and playmysong
Furthermore, recent projects suggest a framework to share the role of the DJ in a bar or party between the participants (\cite{web:shaw}).
Using those systems participants can choose the music by themselves and the ``playlist'' is created dynamically by their musical taste of the participants.
Most of those project are implemented as mobile applications and some them even integrate social elements in their projects (\cite{web:playmysong}; \cite{web:lammers}).

\subsection{Technology dependent social networking}

% Silent disco and flash mobs. LBS are out because they are not really relevant!
Silent disco and flash mobs, the conceptual roots of this research, are examples of modern type of social behavior that rely on the fast growth of social media. My proposal operates in the context of silent disco party and is inspired by flash mobs in the usage of new technologies to facilitate creative and artistic social interactions.

Silent disco is the phenomenon of partying where the music is heard through headphones instead of loudspeakers.
The origins of the phenomena are unclear, but it began to be an ordinary way of partying toward the beginning of 2000's (``\citetitle{wiki:silentdisco}'').
The new phenomena already changed the possibilities of an ordinary party.
One new possibility was having two DJs spin two completely different sets side by side at the same party where each participant has two channel wireless headphones, and can decide which DJ to listen to (\cite{web:headphonedisco}).
Of course, this is not possible with a regular loudspeaker setup\todo{Add a note about silent disco where every participant come with his own music}.

Flash mobs is the phenomena ``A group of people who assemble suddenly in a public place, perform an unusual and seemingly pointless act for a brief time, then quickly disperse, often for the purposes of entertainment, satire, and artistic expression. Flash mobs are organized via telecommunications, social media, or viral emails'' (``\citetitle{wiki:flashmob}'')\todo{rephrase, DON'T USE WIKI!}.
With regards to the definition above, with emphasis on the way flash mobs are organized and executed, different researchers see the flash mobs as a significant event in the history of mobile communication (\cite{nicholson05}).
On the other hand, a recent study by Brejzek suggests that a potential for artistic intent inherently exists in the flash mob phenomenon (\cite*{brejzek10}).

\subsection{Social effects of music}

% Based, as suggested by Avi, on ``The social psychology of music -- David Hargreaves''.
% Consider also "The Do Re Mi’s of Everyday Life: The Structure and Personality Correlates of Music Preferences" by Rentfrow and Gosling.
% and: Hargreaves, David J., Dorothy Miell, and Raymond AR MacDonald. "What are musical identities, and why are they important." Musical identities (2002): 1-20.

\todo{the following is only a sketch, write it!}

Some introduction.
Music as a way for communication.

Music social function.
The influence on individual self identity and interpersonal relationships as described by Hargreaves and North.

Classification of personality dimensions, self-views and cognitive abilities according to musical preferences as suggested by Rentfrow and Gosling.

% Don't forget to add conclusions from those researches into the roadmap section.

\subsection{Indoor positioning systems}

The system I proposed will require the ability to locate the positioning of the users within an indoor environment.

Today, the usage of outdoor positioning systems is unquestioned and achieved mainly by the General Positioning System (GPS) which is available in almost any modern phone.
On the other hand, IPS has not yet standardized and therefor it is are still not available to the average user.\todo{refs needed for the whole paragraph}

Recent researches state that the two most favored IPS technologies are WiFi ``fingerprinting'' and real-time locating systems (RTLS) (\cite{web:harrop}).
WiFi ``fingerprinting'' is the technique where the analysis of received signal strength indicators (RSSI) is done once for different points inside the building to generate an emission map.
After processing such a map any WiFi enable device can compare RSSI with the map to reveal its positioning (\cite{chen}).
On the other hand, RTLS use access points to locate the WiFi devices in their range and determine their positioning positioning through triangulation (\cite{liu}).
Those two methods can enhance accuracy by applying inertial navigation using the device accelerometer, gyroscope and other available sensors.\todo{check whether using \cite{liu} for the whole paragraph}