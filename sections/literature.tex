\section{Literature review}

My research is located at the intersection of three main subjects:
\begin{enumerate}
	\item \st{music technology and} Interactive music\st{al} systems\st{ (including mobile)},
	\item \new{modern, technology dependent, social phenomena}
	\item \new{and social effects of music}
\end{enumerate}
\st{social interaction, and location based services}. In this review I will shortly describe each subject, elucidating only the aspects relevant to my research, while exploring some of the recent studies showing possible connections between these subjects. \new{I will end this review by presenting how those different fields combines together, and stand on the importance of each one of them to my work.}

\subsection{Interactive music systems}

% Only recent projects and researches (as opposed to the introduction). Emphasis on the ``interactive consumption'' trend. Including brief description of modern, social / interactive way of music consumption (AKA Spotify, project ADA, Smule apps and RjDj, DistributedDJ, Yoni Bloch startup etc.).

\new{According to the Oxford Dictionary to ``interact'' is to ``act in such a way as to have an effect on each other'' (}\todo{add ref}\new{). ``Act'' and ``affect'' are visibly the main concepts of interactivity. In the field of interactive music systems these ideas may be implemented in different ways, \st{. It could be playing a music instrument (ref), interacting with an installation (Visnjic, 2010), distribute the control over the music to different people (DistributedDJ) and more. Interactive music systems}} ranging from \new{interactive sound installations, where participants interact with physical objects around them to affect the sound} \st{very basic types of controllers that use some physical input to generate music} (Visnjic, 2010\st{; Silver, 2011; Flanagan, 2009}, \href{http://createdigitalmusic.com/2012/10/at-musicmakers-experiencing-music-through-design-as-community-of-doers-collaborates-listen-watch/}{\todo{finish another ref}}) to much complex approaches \new{that, as manifested by MIT Opera of the Future research group,} aim\st{ing}\new{s} to ``explore concepts and techniques to help advance the future of musical composition, performance, learning, and expression'' (\href{http://www.media.mit.edu/research/groups/opera-future}{``Opera of the future''})\st{; Palmer, 2009.)}.

% The rest of the subsection is problematic. Write it again. State more relevant works and keep it simple.

\st{In addition to systems that enable the user to express himself through technological means, there are also studies that explore ways of human-machine musical interaction, where the machine has autonomy to react to musical situation (Assayag, 2006; Blackwell, 2002).}

\st{In addition to studies in the fields of music interaction, social interaction and location-based services we also find research exploring the boundaries between these fields.}

\st{The ability to help the blind with an audio-based augmented reality system that converts pictures to sound was one of the first suggestions to use audio as the virtual layer of augmented reality system (Meijer, 2013). An automated tour guide concept was also proposed, with the ability to add another layer of information based on the user location (Bederson, 1995).}

\st{Later, this pragmatic use of audio, location and interaction started to be used for artistic means: there are projects where sound is virtually placed in the space and can interact with users (``Audio Space''; Kirn, 2011). The growth of modern mobile devices brings interesting audio-based augmented reality applications (``AutoRap by Smule'',''RjDj''). Additionally, the technological social behavior brings ways to create shared music for events by distributing the role of the DJ to many users (Shaw, 2011; Shapira and Dubnov).}

\subsection{Modern, technology dependent, social phenomena}

% Silent disco and flash mobs. LBS are totally removed from the research because it is not really relevant!

The fast growth of social media produced modern types of social behavior. When the modern mobile devices replaced the old mobile communication devices they became more than just a communication tool (Srivastava, 2005). Two particular phenomena that emerged from that situation that are relevant to my study are silent disco, and flash mobs.

Silent disco \st{/ rave are different names for the same} \new{is the} phenomenon of partying where the music is heard through headphones instead of loudspeakers. The origins of the phenomena is unclear, but it began to be an ordinary way of partying toward the beginning of 2000's (\href{http://en.wikipedia.org/wiki/Silent_disco}{``Silent disco''}). The new phenomena already changed the possibilities of an ordinary party. One new possibility was having two DJs spin two completely different sets side by side at the same party where each participant has two channel wireless headphones, and can decide which DJ to listen to (\href{http://headphonedisco.com/about.php}{``About headphone disco''}). Needless to say, this is not possible with a regular loudspeaker setup.

Another phenomenon relevant to my research is that of flash mobs: \st{``A public gathering of strangers and acquaintances organized via email and texting. Once gathered, flash mobbers perform a quirky stunt and then quickly disperse. This social act shone briefly and brilliantly in cities around the world in summer 2003 and is believed to represent a significant moment in the history of mobile communication'' (Nicholson, 2005).} \new{``A group of people who assemble suddenly in a public place, perform an unusual and seemingly pointless act for a brief time, then quickly disperse, often for the purposes of entertainment, satire, and artistic expression. Flash mobs are organized via telecommunications, social media, or viral emails''} (\href{http://en.wikipedia.org/wiki/Flash_mob}{``Flash mob''}). \new{With regards to the definition above, with emphasis on the way flash mobs are organized and executed, different researchers see the flash mobs as a significant event in the history of mobile communication (Nicholson, 2005).} \st{Flash mobs today make use of new technology to enable distributed parties when the music is determined by the flash mob participants in real time (``Decentralized Dance Party'')}\todo{seems that it's more relevant to silent disco}. \st{Recent studies proposed that a potential for artistic intent and social critique inherently exists in the flash mob's appropriation of the city as a scenographic, performative space (Brejzek, 2010)}. \new{On the other hand, a recent study by Brejzek suggests that a potential for artistic intent inherently exists in the flash mob phenomenon (2010).}

\st{From the different fields of study related to music technology, the most important area for my research is that relating to music interaction systems and audio based augmented reality systems.}

\st{Today, most social networks combine LBS in their platform. Examples are Foursquare, Facebook and their purchase of Gowalla (Laughlin, 2011), Google plus ``Check in'' and more. Recent studies of this service market also show the growth of the usage of LBS among users (Zickuhr and Smith, 2011; Miltsch, 2010), and of mobile users who don't use mobile LBS, 60\% are willing to consider using it (Lawrence, 2012).}

\subsection{Social effects of music}

% Based, as suggested by Avi, on ``The social psychology of music -- David Hargreaves''.

\subsection{Roadmap}

% Describe the system and its connection with concepts mentioned before. Everything bellow is new!

The system developed in this research has been inspired by the concepts of interactive music systems, with emphasis on interactive music consumption and key projects as shown in the last chapter. In addition it sees both silent disco and flash mobs as significant conceptional roots. Whereas silent disco will be referred as a context for the system existent, flash mobs are associated as a social phenomena that use new technologies to facilitate creative, even artistic social behavior.

% Explain what the system does.

Finally, in order to evaluate the social effect of this system I will try to answer the research question: \emph{Does the system elaborate social interaction between participants in an interactive silent disco party?}