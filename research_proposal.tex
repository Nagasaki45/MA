\documentclass[a4paper,11pt]{article}
\usepackage{soul}
\usepackage{hyperref}
\usepackage{todonotes}

\newcommand{\new}[1]{\textcolor{blue}{#1}}

\title{Audio-Only Augmented Reality System for\\Social Interaction} 
\author{Tom Gurion}

\begin{document}
\maketitle

\section{Introduction and framework}

% Covering the history of music technology and interactive music system.
% Don't enter into mobile devices, social networks, indoor positioning or location based services, keep them to the literature review or to the system development chapter.
% Add a paragraph for short survey of AR.

In the past 60 years the development of new technology has fundamentally transformed music creation and consumption (Winkler, 2001: \todo{what page?}). Today, interactive music systems are used in many different contexts ranging from instrument design to socially interactive installations (Drummond, 2009). In this work I will focus on the new possibilities that real-time social interaction\new{, with the help of modern mobile technology,} can offer to \st{interactive} music consumption. \st{I will describe the history of music interactive systems and show different relevant works by other researchers.} \new{I will start with an historic review of the field of interactive music systems with emphasis on key events and important projects.}

\st{Furthermore, the field of music interactive systems is closely related to other fields of research that will be important for my work as well. Among these are social interaction and social media (Kaplan and Haenlein, 2010), augmented reality (Azuma, 1997), positioning technology (``Positioning technology''), location based services (Schiller and Voisard, 2004) and more.}

Since the beginning of the exploration \new{in the field} of \new{interactive} music \st{interactive} system\new{s} in the nineteen-sixties, different researchers and composers created systems that were able to interact with performer in a live situation. First examples of this kind of \st{interactive music} \st{are} \new{interaction} \new{were} Gordon Mumma's Hornpipe, and Morton Subotnick's Touch (\href{http://blog.lib.umn.edu/geers001/emusic/14_assig_ComposingInteractiveMusicCh1-2.pdf}{Winkler, 2001: 12})\new{, where specially designed electronic systems alter the input from the performer in some nondeterministic, yet musically defined, manner}. Later, programming languages designed for musical applications began to be developed, such as GROOVE by Max Mathews and Richard Moore at Bell labs (Mathews and Moore, 1970). \new{Using those languages musicians was able to create system like those of Mumma and Subotnick programmatically}.

In the nineteen-eighties, a group of musical instruments manufacturers agreed on standard method for sending and receiving musical information digitally, establishing \new{the }MIDI \new{protocol} as a universal standard (\href{http://www.insidetechnology360.com/index.php/the-history-of-midi-8862/}{Quinn, 2010}). The standardization of sending and receiving information combined with the emergence of personal computers enabled the creation of \st{commercial and open source} \new{modern} programing languages for musical application\new{s}\st{, the most important one today is}. Max/MSP (\href{http://cycling74.com/whatismax/}{\st{``What is Max?''}}) \new{which began its development} by Miller Puckette \st{from IRCAM, Paris} in 1986 \new{may be a good example} (\href{http://blog.lib.umn.edu/geers001/emusic/14_assig_ComposingInteractiveMusicCh1-2.pdf}{Winkler, 2001: 16}). \new{As opposed to early programming languages as GROOVE, most of those personal computer based languages not only still exist today but also keep progressing and new music oriented programing languages are written every year} (\todo{reference ChucK}).\st{ One of the main uses of these new programming languages was to compose interactive music that can manipulate sound during performance in real-time.}

\new{But the proliferation of the personal computer not only drive the creation of programing languages for music application. In the nineteen-nineties} \st{Due to new opportunities in digital signal processing technologies in the nineteen-nineties, and the proliferation of personal computers,} digital audio workstations (DAW) became a common alternative to analog recording \new{equipment} \st{systems} in \st{recording} studios (\todo{find reference}). Later, with the arrival of the VST standard (\href{http://www.steinberg.net/en/company/technologies.html}{``Steinberg technologies''}) \st{and Ableton Live (}\href{https://www.ableton.com/en/live/}{\st{``What is Live?''}}), computers became \new{even more essential tool for music production.} \st{the main tool for audio manipulation in real time.} \new{Those changes also find their way to the stage with the emergence of live performance oriented DAWs as Ableton Live (}\href{https://www.ableton.com/en/live/}{\new{``What is Live?''}}\new{) and software based DJ setups in late nineteen-nineties.} 

\new{Another key event in the history of interactive music systems is the appearance of the Arduino platform in 2006. The Arduino is an easy to use hardware and software package, ``intended for artists, designers, hobbyists and anyone interested in creating interactive objects or environments.'' (}\todo{Arduino homepage ref here}\new{) Until its appearance the only way for a musician to interact with music creation software was through audio and MIDI. But from then }\st{When microcontrollers like the Arduino platform (``Arduino'') began to appear in mid 2000's} a lot of new ways of controlling audio started to be possible, mostly by translating physical properties of the space into sound.

Today, one can find the cutting edge research in music interactive systems and the exploration of new ways of music creation and music expression in ambitious projects like the Multimodal Brain Orchestra (\href{http://specs.upf.edu/installation/2025}{\todo{reference madrid SPECS group}}) and Urban Musical Games (\href{http://imtr.ircam.fr/imtr/IRCAM_Real-Time_Musical_Interactions}{``IRCAM real-time musical interactions''}).

\new{On the other hand, }with the emergence of modern mobile devices, interactive music systems have become accessible to non-musicians (for example AutoRap by Smule [4], and RjDj [5]) as well as facilitating a shared process of music creation between different users in social context [6-7].

\new{Another key concept with high relevance to my work is Augmented Reality (AR). According to Azuma, AR systems integrate 3-D virtual objects into a 3-D real environment in real time (1997). Given this necessarily visual definition it is clear that there is no way to accomplish ``Audio-Only'' AR system. Later in his article Azuma states that ``AR enhances a user's perception of and interaction with the real world'', a statement which lighten the subject in much broader manner and open the door to enhance ones perception of and interaction with the world with much more than just visual information.}

\section{Research targets}

\new{In this research I will focus on two main targets:}
\begin{enumerate}
	\item To propose and implement an audio-only augmented reality system for social interaction. \new{Using the system, participants will be able to interact with one another as well as with system's components and affect the structure of the music in the virtual space.} \st{This system will let a user hear music generated by his or her physical location in space. In addition, the social interaction between users will affect the musical structure of the space in real time. As a development over mobile social interactive ideas like flash mobs and silent disco, the system will be developed for Android OS. Relative positioning in space will be determined using Bluetooth technology. Music will be created especially for the system and will be heard through headphones by the users.}
	\item \new{To evaluate the social effects of the system usage }\new{\st{on participants}}\new{ in the context of a silent rave party. }\st{In addition, I will apply different methods for system evaluation such as questionnaires and participants positioning tracking during system usage. Using those methods it will be possible to gain better understanding of the user's social interaction using my system.}
\end{enumerate}
\st{More generally, my research explores the potential for a new way of music consumption through mobile devices, where the consumer's social interaction is the main influence over actually heard music. Finally, my research explores the ways in which this interactive mode of music consumption can create new musical materials.}

\section{Literature review}

My research is located at the intersection of three main subjects:
\begin{enumerate}
	\item \st{music technology and} Interactive music\st{al} systems\st{ (including mobile)},
	\item \new{modern, technology dependent, social phenomena}
	\item \new{and social effects of music}
\end{enumerate}
\st{social interaction, and location based services}. In this review I will shortly describe each subject, elucidating only the aspects relevant to my research, while exploring some of the recent studies showing possible connections between these subjects. \new{I will end this review by presenting how those different fields combines together, and stand on the importance of each one of them to my work.}

\subsection{Interactive music systems}

% Only recent projects and researches (as opposed to the introduction). Emphasis on the ``interactive consumption'' trend. Including brief description of modern, social / interactive way of music consumption (AKA Spotify, project ADA, Smule apps and RjDj, DistributedDJ, Yoni Bloch startup etc.).

\new{According to the Oxford Dictionary to ``interact'' is to ``act in such a way as to have an effect on each other'' (}\todo{add ref}\new{). ``Act'' and ``affect'' are visibly the main concepts of interactivity. In the field of interactive music systems these ideas may be implemented in different ways, \st{. It could be playing a music instrument (ref), interacting with an installation (Visnjic, 2010), distribute the control over the music to different people (DistributedDJ) and more. Interactive music systems}} ranging from \new{interactive sound installations, where participants interact with physical objects around them to affect the sound} \st{very basic types of controllers that use some physical input to generate music} (Visnjic, 2010\st{; Silver, 2011; Flanagan, 2009}, \href{http://createdigitalmusic.com/2012/10/at-musicmakers-experiencing-music-through-design-as-community-of-doers-collaborates-listen-watch/}{\todo{finish another ref}}) to much complex approaches \new{that, as manifested by MIT Opera of the Future research group,} aim\st{ing}\new{s} to ``explore concepts and techniques to help advance the future of musical composition, performance, learning, and expression'' (\href{http://www.media.mit.edu/research/groups/opera-future}{``Opera of the future''})\st{; Palmer, 2009.)}.

% The rest of the subsection is problematic. Write it again. State more relevant works and keep it simple.

\st{In addition to systems that enable the user to express himself through technological means, there are also studies that explore ways of human-machine musical interaction, where the machine has autonomy to react to musical situation (Assayag, 2006; Blackwell, 2002).}

\st{In addition to studies in the fields of music interaction, social interaction and location-based services we also find research exploring the boundaries between these fields.}

\st{The ability to help the blind with an audio-based augmented reality system that converts pictures to sound was one of the first suggestions to use audio as the virtual layer of augmented reality system (Meijer, 2013). An automated tour guide concept was also proposed, with the ability to add another layer of information based on the user location (Bederson, 1995).}

\st{Later, this pragmatic use of audio, location and interaction started to be used for artistic means: there are projects where sound is virtually placed in the space and can interact with users (``Audio Space''; Kirn, 2011). The growth of modern mobile devices brings interesting audio-based augmented reality applications (``AutoRap by Smule'',''RjDj''). Additionally, the technological social behavior brings ways to create shared music for events by distributing the role of the DJ to many users (Shaw, 2011; Shapira and Dubnov).}

\subsection{Modern, technology dependent, social phenomena}

% Silent disco and flash mobs. LBS are totally removed from the research because it is not really relevant!

The fast growth of social media produced modern types of social behavior. When the modern mobile devices replaced the old mobile communication devices they became more than just a communication tool (Srivastava, 2005). Two particular phenomena that emerged from that situation that are relevant to my study are silent disco, and flash mobs.

Silent disco \st{/ rave are different names for the same} \new{is the} phenomenon of partying where the music is heard through headphones instead of loudspeakers. The origins of the phenomena is unclear, but it began to be an ordinary way of partying toward the beginning of 2000's (\href{http://en.wikipedia.org/wiki/Silent_disco}{``Silent disco''}). The new phenomena already changed the possibilities of an ordinary party. One new possibility was having two DJs spin two completely different sets side by side at the same party where each participant has two channel wireless headphones, and can decide which DJ to listen to (\href{http://headphonedisco.com/about.php}{``About headphone disco''}). Needless to say, this is not possible with a regular loudspeaker setup.

Another phenomenon relevant to my research is that of flash mobs: \st{``A public gathering of strangers and acquaintances organized via email and texting. Once gathered, flash mobbers perform a quirky stunt and then quickly disperse. This social act shone briefly and brilliantly in cities around the world in summer 2003 and is believed to represent a significant moment in the history of mobile communication'' (Nicholson, 2005).} \new{``A group of people who assemble suddenly in a public place, perform an unusual and seemingly pointless act for a brief time, then quickly disperse, often for the purposes of entertainment, satire, and artistic expression. Flash mobs are organized via telecommunications, social media, or viral emails''} (\href{http://en.wikipedia.org/wiki/Flash_mob}{``Flash mob''}). \new{With regards to the definition above, with emphasis on the way flash mobs are organized and executed, different researchers see the flash mobs as a significant event in the history of mobile communication (Nicholson, 2005).} \st{Flash mobs today make use of new technology to enable distributed parties when the music is determined by the flash mob participants in real time (``Decentralized Dance Party'')}\todo{seems that it's more relevant to silent disco}. \st{Recent studies proposed that a potential for artistic intent and social critique inherently exists in the flash mob's appropriation of the city as a scenographic, performative space (Brejzek, 2010)}. \new{On the other hand, a recent study by Brejzek suggests that a potential for artistic intent inherently exists in the flash mob phenomenon (2010).}

\st{From the different fields of study related to music technology, the most important area for my research is that relating to music interaction systems and audio based augmented reality systems.}

\st{Today, most social networks combine LBS in their platform. Examples are Foursquare, Facebook and their purchase of Gowalla (Laughlin, 2011), Google plus ``Check in'' and more. Recent studies of this service market also show the growth of the usage of LBS among users (Zickuhr and Smith, 2011; Miltsch, 2010), and of mobile users who don't use mobile LBS, 60\% are willing to consider using it (Lawrence, 2012).}

\subsection{Social effects of music}

% Based, as suggested by Avi, on ``The social psychology of music -- David Hargreaves''.

\subsection{Roadmap}

% Describe the system and its connection with concepts mentioned before. Everything bellow is new!

The system developed in this research has been inspired by the concepts of interactive music systems, with emphasis on interactive music consumption and key projects as shown in the last chapter. In addition it sees both silent disco and flash mobs as significant conceptional roots. Whereas silent disco will be referred as a context for the system existent, flash mobs are associated as a social phenomena that use new technologies to facilitate creative, even artistic social behavior.

% Explain what the system does.

Finally, in order to evaluate the social effect of this system I will try to answer the research question: \emph{Does the system elaborate social interaction between participants in an interactive silent disco party?}

\section{Research methods}

\new{This chapter, as well as the next one, will be divided into two main sections, the first describing the system development and the second the system evaluation.}

\subsection{System development}

\subsubsection{Mobile and Android}

% Deciding to go mobile, including all the information from the literature review about the growth of social use of mobile devices etc.

\new{Recent studies state that today's mobile phone ``has become such an important aspect of a user's daily life that it has moved from being a mere 'technological object' to a key 'social object''' (Srivastava, 2005), and as such it has a significant importance in shaping today's society. As denoted in chapter 3.4 this research aims to follow social phenomena that use those abilities of mobile phone and therefore will be implemented for mobile.}

The system will be developed on Android OS (\todo{ref}). Choosing Android OS as the platform for my research has two main advantages:

\begin{enumerate}

	\item The Android system is a growing mobile system which controls most of the market share today ("Mobile operating system").

	\item By developing application for Android I have access to underlying Bluetooth properties such as received signal strength indicator (RSSI) essential for my implementation of the system \new{as lay out in the next section}.

\end{enumerate}

\subsubsection{Bluetooth based relative indoor positioning (BBRIP) system}

% Explain here why I've decided to implement my own indoor positioning system. For HCI it was a good idea to push it to the front, for BI music department I'm not so sure...

Today, the usage of outdoor positioning systems \st{(mainly GPS)} is unquestioned \new{and achieved mainly by the GPS standard available in almost any modern phone}. \new{On the other hand, indoor positioning systems (IPS) are not yet standardized and therefor} \st{but the opportunity of indoor location awareness technology is} not \st{yet possible} \new{available} to the \st{end} \new{average} user.

% Rewrite the next two paragraphs.

\st{There are many difficulties in the establishment of indoor positioning system (Borenović and Nešković, 2010). One of the ways to enable IPS is through Bluetooth technology. As a low power consumption technology widely supported by mobile devices, Bluetooth has a potential to be used for indoor positioning (Pei et al. 2010)}.

\st{Another option for indoor positioning system is to use relative positioning. Research has shown the advantages of ultrasonic relative IPS because to the fact that it doesn’t need infrastructure, unlike most of the above mentioned technologies (Hazas, 2005).}

\subsubsection{libpd}

\subsection{System evaluation}

% The following, as described in the meeting summary.

\subsubsection{Participants}

\subsubsection{Measures}

\subsubsection{Procedure}

\section{Initial results}

\subsection{System development}

\subsubsection{Implementation of the BBRIP system}

\subsubsection{ScenePlayer Plus}

\subsection{System evaluation}

% All of the pilot information. Including its separate participants, measures and procedure. It's different from what we will describe in the new ``system evaluation'' chapter above.

\section{Appendix}

% Questionnaires: Standardized musical background survey, modified system usability scale (SUS), Interaction survey.

\section{Bibliography}

\end{document}